\documentclass[a4paper]{article}
\usepackage[utf8]{inputenc}
\usepackage[italian]{babel}

\def\CC{{C\nolinebreak[4]\hspace{-.05em}\raisebox{.4ex}{\tiny\bf ++}}}

\title{\textbf{Jump king}\\
\vspace{0.2cm}\normalsize Relazione del progetto per il corso di Programmazione\\
\normalsize Università di Bologna
}

\author{
  E. Argonni,
  O. Ayache,
  L. Peronese,
  S. Musiani
}

\date{2022 - 2023}

\begin{document}

\maketitle

\section{Introduzione}
\emph{Jump king} è un gioco platform a scorrimento verticale illimitato 
in grafica ASCII completamente realizzato in \CC. L'esecuzione del gioco 
avviene interamente all'interno di un terminale, dove è possibile eseguire tutte
le azioni necessarie attraverso la tastiera. Sono presenti diversi tipi di 
nemici e armi, queste ultime acquistabili in un apposito market tramite le 
monete raccolte durante la partita. Essendo centrale nel gameplay la meccanica 
di salto, è stata implementata quest'ultima in modo che si possa scegliere, in 
funzione del tempo di pressione dell'apposito tasto, la forza del salto. È stato
inoltre previsto un pannello per calibrare la tastiera in modo che l'esperienza
di gioco sia il più simile possibile su tutti i dispositivi in cui viene 
giocato.

\section{Spartizione dei ruoli}
Essendo il progetto sviluppato su più classi, è stato possibile lavorare 
parallelamente suddividendo il lavoro tra tutti i componenti del gruppo. Nello 
specifico la suddivisione è stata la seguente:
\begin{itemize}
  \item \emph{E. Argonni}: Implementazione della classe \texttt{Game}, ovvero la 
    classe più importante di tutto il gioco in quanto si interfaccia con tutte 
    le restanti classi. Grazie ad essa infatti c'è una coesione tra tutte le 
    porzioni di codice sviluppate separatamente. Al suo interno è possibile 
    trovare la funzione che contiene il loop principale in cui si  svolge una 
    partita. \texttt{Menu} è la classe adibita alla gestione del menu iniziale 
    di gioco ed è inoltre responsabile dell'animazione a inizio gioco.
    \texttt{Save} è classe che si occupa di interagire con l'utente quando è 
    necessario salvare una partita. Si occupa però solo dell'aspetto grafico, 
    mentre la classe che esegue l'effettivo salvataggio su file è differente. 
    La \texttt{Settings} invece si occupa delle impostazioni: personalizzazzione 
    dei tasti di gioco, calibrazione della tastiera e altre impostazioni 
    specifiche di minori. Infine la classe \texttt{Statistcs} che si occupa di 
    tenere traccia della vita, delle moenete e in generale delle statiche del 
    giocatore. (RIGHE: 1196)

  \item \emph{O. Ayache}: Implementazione delle classi \texttt{Screen} e 
    \texttt{Draw} che hanno il compito di nascondere completamente tutte le 
    funzioni di \texttt{ncurses}\footnote{\texttt{Ncurses} è la libreria esterna usata per gestire la grafica su terminale del gioco} in due semplici e comode 
    classi in modo che tutte le classi avente necessità di disegnare sullo 
    schermo potessero semplicemente interfacciarsi a loro, senza usare 
    direttamente la libreria \texttt{ncurses}. Questo ha permesso di aggiungere 
    un livello di astrazione maggiore che nascondesse l'implementazione a basso 
    livello della libreria. La classe \texttt{Entity} è una generica entità nel
    gioco che grazie all'eredità fornita dal paradigma OOP è la classe padre di
    \texttt{Enemy}, \texttt{Coin} e \texttt{Bullet} che da sole sono già 
    autoesplicative. La classe \texttt{Market} si occupa di disegnare tutto il 
    menu relativo al negozio e di fornire all'utente un metodo facile per 
    l'acquisto di armi e abilità. Qeueste ultime sono rappresentate grazie a
    delle classi apposite: \texttt{Gun} e \texttt{Ability}. Infine la classe 
    \texttt{Events} si occupa della gestione delle e della loro integrazione con
    la partita. (RIGHE: 920)

  \item L. Peronese: Implementazione delle mappe tramite la classe \texttt{Map}
    insieme alle classi \texttt{Chunk} e \texttt{Platform}. Queste ultime sono
    frammenti più piccoli che composti formano una mappa infinita  e facilmente 
    utilizzabile attraverso la classe apposita. Il salvataggio delle partite, 
    delle impostazioni, come i tasti personalizzati, e la calibrazione è 
    delegato alla classe \texttt{File}. Questa classe si occupa della effettiva 
    scrittura su disco ed è quindi invisibile all'utente in quanto chiamata da 
    altre classi. Inoltre si occupa di leggere i salvataggi quando 
    il gioco viene riaperto o è necessario caricare una partita precedentemente 
    salvata. La classe \texttt{Random} viene usata come interfaccia dalle altre
    classi in qualsisi momento sia necessario generare numeri casuali per avere
    un luogo unico dove vengono gestiti i \emph{seed}. Per fare debug, in quanto
    lo \texttt{stdout} è occupato dalla grafica del gioco, è stata scritta la 
    libreria di funzioni \texttt{logs} in quanto permettono di scrivere su file 
    stringhe arbitrarie. (RIGHE: 834)

  \item S. Musiani:
\end{itemize}

\section{Scelte implementative}

\end{document}
