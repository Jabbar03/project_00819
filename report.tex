\documentclass[a4paper]{article}
\usepackage[utf8]{inputenc}
\usepackage[italian]{babel}

\def\CC{{C\nolinebreak[4]\hspace{-.05em}\raisebox{.4ex}{\tiny\bf ++}}}

\title{\textbf{Jump king}\\
\vspace{0.2cm}\normalsize Relazione del progetto per il corso di Programmazione\\
\normalsize Università di Bologna
}

\author{
  E. Argonni,
  O. Ayache,
  L. Peronese,
  S. Musiani
}

\date{2022 - 2023}

\begin{document}

\maketitle

\section{Introduzione}
\emph{Jump king} è un gioco platform a scorrimento verticale illimitato 
in grafica ASCII completamente realizzato in \CC. L'esecuzione del gioco 
avviene interamente all'interno di un terminale, dove è possibile eseguire tutte
le azioni necessarie attraverso la tastiera. Sono presenti diversi tipi di 
nemici e armi, queste ultime acquistabili in un apposito market tramite le 
monete raccolte durante la partita. Essendo centrale nel gameplay la meccanica 
di salto, è stata implementata quest'ultima in modo che si possa scegliere, in 
funzione del tempo di pressione dell'apposito tasto, la forza del salto. È stato
inoltre previsto un pannello per calibrare la tastiera in modo che l'esperienza
di gioco sia il più simile possibile su tutti i dispositivi in cui viene 
giocato.

\section{Spartizione dei ruoli}
Essendo il progetto sviluppato su più classi, è stato possibile lavorare 
parallelamente suddividendo il lavoro tra tutti i componenti del gruppo. Nello 
specifico la suddivisione è stata la seguente:
\begin{itemize}
  \item \emph{E. Argonni}: Implementazione della classe \texttt{Game}, ovvero la 
    classe più importante di tutto il gioco in quanto si interfaccia con tutte 
    le restanti classi. Grazie a essa infatti c'è una coesione tra tutte le 
    porzioni di codice sviluppate separatamente. Al suo interno è possibile 
    trovare la funzione che contiene il loop principale in cui si  svolge una 
    partita. \texttt{Menu} è la classe adibita alla gestione del menu iniziale 
    di gioco ed è inoltre responsabile dell'animazione a inizio gioco.
    \texttt{Save} è classe che si occupa di interagire con l'utente quando è 
    necessario salvare una partita. Si occupa però solo dell'aspetto grafico, 
    mentre la classe che esegue l'effettivo salvataggio su file è differente. 
    La \texttt{Settings} invece si occupa delle impostazioni: personalizzazione 
    dei tasti di gioco, calibrazione della tastiera e altre impostazioni 
    specifiche minori. Infine la classe \texttt{Statistcs} che si occupa di 
    tenere traccia della vita, delle monete e in generale delle statiche del 
    giocatore. (RIGHE: 1317)

  \item \emph{O. Ayache}: Implementazione delle classi \texttt{Screen} e 
    \texttt{Draw} che hanno il compito di nascondere completamente tutte le 
    funzioni di \texttt{ncurses}\footnote{\texttt{Ncurses} è la libreria esterna 
    usata per gestire la grafica su terminale del gioco} in due semplici e 
    comode classi in modo che tutte le classi avente necessità di disegnare 
    sullo schermo potessero semplicemente interfacciarsi a loro, senza usare 
    direttamente la libreria \texttt{ncurses}. Questo ha permesso di aggiungere 
    un livello di astrazione maggiore che nascondesse l'implementazione a basso 
    livello della libreria. La classe \texttt{Entity} è una generica entità nel
    gioco che grazie all'eredità fornita dal paradigma OOP è la classe padre di
    \texttt{Enemy}, \texttt{Coin} e \texttt{Bullet} che da sole sono già 
    autoesplicative. La classe \texttt{Market} si occupa di disegnare tutto il 
    menu relativo al negozio e di fornire all'utente un metodo facile per 
    l'acquisto di armi e abilità. Queste ultime sono rappresentate grazie a
    delle classi apposite: \texttt{Gun} e \texttt{Ability}. Infine la classe 
    \texttt{Events} si occupa della gestione delle e della loro integrazione con
    la partita. (RIGHE: 920)

  \item \emph{L. Peronese}: Implementazione delle mappe tramite la classe 
    \texttt{Map} insieme alle classi \texttt{Chunk} e \texttt{Platform}. Queste 
    ultime sono frammenti più piccoli che composti formano una mappa infinita  e 
    facilmente utilizzabile attraverso la classe apposita. Il salvataggio delle 
    partite, delle impostazioni, come i tasti personalizzati, e la calibrazione 
    è delegato alla classe \texttt{File}. Questa classe si occupa della 
    effettiva scrittura su disco ed è quindi invisibile all'utente in quanto 
    chiamata da altre classi. Inoltre si occupa di leggere i salvataggi quando 
    il gioco viene riaperto o è necessario caricare una partita precedentemente 
    salvata. La classe \texttt{Random} viene usata come interfaccia dalle altre
    classi in qualsiasi momento sia necessario generare numeri casuali per avere
    un luogo unico dove vengono gestiti i \emph{seed}. Per fare debug, in quanto
    lo \texttt{stdout} è occupato dalla grafica del gioco, è stata scritta la 
    libreria di funzioni \texttt{logs} che permettono di scrivere su file 
    stringhe arbitrarie. (RIGHE: 834)

  \item \emph{S. Musiani}: Implementazione della libraria di fisica per la 
    simulazione di oggetti, gravità, velocità e salti. Distinta del namespace 
    \texttt{phy} comprende le seguenti classi: \texttt{Point} per rappresentare 
    un punto su un piano bidimensionale; \texttt{PrecisesPoint} per 
    rappresentare un punto su un piano bidimensionale il più precisamente 
    possibile, cioè senza approssimazioni; \texttt{Vector} per la 
    rappresentazione di un vettore; \texttt{Body} per la simulazione di un 
    corpo che ha una posizione, una velocità e una accelerazione. Sono inoltre 
    presenti delle costanti nel file \texttt{constants.hpp}, come 
    l'accelerazione della gravità in gioco e delle funzioni per simulare le 
    collisioni tra oggetti nel file \texttt{collisions.cpp}. La scrittura della 
    classe \texttt{Manager} che tiene traccia e gestisce tutte le entità nel 
    gioco come monete, proiettili e nemici. Tiene traccia della loro morte, dei 
    loro movimenti e dei loro comportamenti, anche in relazione al giocatore. 
    Comunica inoltre con la classe per il salvataggio su file in modo da 
    memorizzare lo stato delle entità efficientemente.

    Essendo da specifiche proibito l'utilizzo della standar library di \CC\ e in
    particolare tutte le classi che facilitano l'allocazione di memoria e la 
    gestione di essa, sono state re-implementate le classi \texttt{string} e 
    \texttt{vector} in particolare entrambi sotto il namespace \texttt{nostd}. 
    Esse fanno quello che ci si aspetta e sono una implementazione personale 
    delle classi di libreria. La classe \texttt{string} è stata realizzata con 
    \emph{L. Peronese}. (RIGHE: 1436)
\end{itemize}

\section{Scelte implementative}
\subsection{Screen e Draw}
Per semplificare il più possibile le operazioni di disegno su schermo e in 
generale tutto quello che riguardasse la grafica sono state create queste due
classi per nascondere completamente la libreria \texttt{ncurses}. In particolare
la classe \texttt{Screen} implementa solo le operazioni base che si possono fare
su uno schermo come inizializzarlo, cancellarlo, impostarne la dimensione ecc.
La classe è creata in modo che tutte le volte che un oggetto \texttt{Screen} 
viene creato automaticamente un nuova finestra di \texttt{ncurses} viene 
allocata.

La classe \texttt{Draw} estende \texttt{Screen} ereditando tutti i metodi base
per gestire lo schermo, ma implementando a sua volta tutti i metodi necessari 
per disegnare. In particolare quelli necessari per scrivere a schermo stringhe, 
interi o semplici caratteri. Vengono inoltre forniti metodi per disegnare figure 
quali linee e rettangoli. Sono presenti anche i metodi per disegnare la mappa, 
il giocatore e tutti i tipi di entità quali nemici, proiettili e monete.

\subsection{Game}
La classe principale del videogioco è \texttt{Game}, la quale gestisce 
l’interazione con le altre classi e il flusso complessivo del gioco. La funzione 
\textsc{run} è la prima a essere chiamata e ha il compito di far partire tutto
il gioco. In particolare esegue il menu principale del gioco che permette 
all’utente di scegliere tra diverse opzioni come giocare, riprendere una partita salvata, accedere alle impostazioni o uscire dal gioco. La funzione 
\textsc{play} racchiude il ciclo principale in cui una partita viene svolta. È
infatti responsabile del progresso di gioco e gestisce l’interazione tra il 
giocatore, le entità, le piattaforme e le statistiche. Tramite i controlli da 
tastiera, il giocatore può muoversi, saltare, sparare e accedere al market. 
La classe gestisce l’aggiornamento della posizione del giocatore e delle entità, la collisione con le piattaforme e consente di mettere in pausa il gioco per 
accedere alle opzioni come le impostazioni, i salvataggi e il market.

\subsection{Menu}
La classe \texttt{Menu} gestisce il menu principale del videogioco e 
contribuisce a creare un’esperienza coinvolgente sin dai primi istanti di avvio. 
È infatti responsabile dell’animazione di intro del gioco e della gestione del 
menu principale, consentendo all’utente di scegliere fra le opzioni disponibili.
La funzione \textsc{getselectedoption} in particolare gestisce la navigazione 
dell’utente all’interno del menu e rende possibile la scelta delle opzioni con 
facilità evidenziando visivamente l’opzione selezionata. 


\subsection{Statistics}
La classe \texttt{Statisitcs} fornisce la struttura per la gestione delle 
statistiche del gioco. Permette all’utente, durante l’esecuzione del gioco, di 
avere un’interfaccia in cui poter controllare in tempo reale le statistiche di 
gioco, il numero di monete disponibili, la quantità di vite possedute, il 
livello a cui si trova e il numero totale di salti effettuati.

\subsection{Settings}
La classe \texttt{Settings} fornisce un'interfaccia intuitiva per la gestione 
delle opzioni di configurazione del gioco. Consente all’utente di personalizzare 
la propria esperienza regolando diversi parametri come la sensibilità del salto, 
il volume della musica e i tasti di controllo del personaggio. Inoltre, è 
possibile effettuare la calibrazione della velocità della propria tastiera con 
cui viene ricaricata la forza di un salto. Grazie all’interfaccia grafica 
l’utente può navigare tra le varie opzioni utilizzando input da tastiera. Una 
volta che l’utente ha effettuato le modifiche desiderate ed è uscito dalle 
impostazioni, modifiche vengono automaticamente salvate e memorizzate per le 
sessioni di gioco successive.

\subsection{Market, Ability, Gun e Events}
Le armi, allo stesso modo delle abilità, sono rappresentate da una classe 
specifica, rispettivamente \texttt{Gun} e \texttt{Ability}, dove vengono
differenziate per i parametri al loro intero. Per esempio il nome, il costo e il 
tempo di ricarica di un'arma. Questo permette di implementare una sola volta 
tutte le funzioni che necessitano di avere in input armi o abilità e fare i 
controlli necessari per distinguerle al loro intero.

Events è la classe che gestisce le abilità e il loro tempo di ricarica. Queste
ultime, infatti, dopo essere acquistate rimangono al giocatore, che però le può
utilizzare solo successivamente a ogni periodo di ricarica. Una barra in alto a
destra nella schermata di gioco mostra se si può utilizzare l'abilità, se si sta
scaricando o se si sta caricando.


\subsection{Random}
Fulcro di questa classe è il \texttt{seed}, ovvero un intero generato 
casualmente all'avvio di una nuova partita e che viene associato a essa; grazie 
a questo dato, la classe \texttt{Random} genera l'ordine dei chunk, la posizione 
delle monete e gli attributi dei nemici utilizzando la generazione di numeri 
casuali inclusa in \CC. Questo metodo ci garantisce la assoluta unicità di ogni 
partita e semplifica molto il salvataggio: non è infatti necessario salvare gli
infiniti valori associati all'ordine dei chunk e alla posizione delle entità, ma 
sarà sufficiente memorizzare il seed associato alla mappa, in quanto le funzioni 
pseudocasuali presenti nel linguaggio generano la stessa sequenza di numeri a 
partire dallo stesso seed.

\subsection{Map, Chunk, Platform}
Queste tre classi si occupano della gestione della mappa di gioco. La classe 
\texttt{Platform} definisce una singola piattaforma nel gioco, che possiede come 
attributi la lunghezza e le coordinate in cui si trova. La classe \texttt{Chunk} 
invece rappresenta una sezione del livello di gioco e come attributo possiede 
un vettore di oggetti \texttt{Platform}; la gestione delle collisioni si 
interfaccia con questa classe per conoscere se in un certo punto c'è una 
piattaforma. La classe \texttt{Map} infine rappresenta il livello generale del 
gioco; qui vengono inizializzati i 10 chunk che vengono selezionati in maniera 
randomica per formare l'infinita mappa di gioco; ogni mappa è caratterizzata da 
un \texttt{seed} univoco.

\subsection{File}
La classe \texttt{File} è preposta ad interfacciarsi con i salvataggi, sia dei 
progressi che delle impostazioni. Nel file \texttt{save.txt} vengono scritti e 
aggiornati tutti i valori della partita attuale, quindi nome, seed, posizione 
nella mappa del giocatore, entità, oggetti acquistati nel market e valori delle 
statistiche; invece, nel file \texttt{settings.txt} vengono memorizzati i tasti
scelti dal giocatore per i comandi di gioco e le sue preferenze riguardo a 
volume e sensibilità. Oltre a svolgere la funzione di scrittura, la classe si 
occupa anche di leggere i file e trasferire i dati alla classe Game per 
ricaricare impostazioni e salvataggi memorizzati in precedenti partite.

\subsection{Fisica}
Tutta la libraria di fisica si ragruppa sotto lo stesso \texttt{namespace} 
denominato \texttt{phy}. Lo scopo di questa libraria è finalizzato soltanto alla
simulazione dei salti del giocatore e al movimento dei proiettili, soprattutto
quelli più complessi come quelli prodotti dal \textit{lancia granate}. Sono
quindi presenti solo gli elementi necessari a simulare il movimento di un punto
nello spazio dotato di posizione, velocità e accelerazione. La posizione in 
particolare è rappresentata dalla classe \texttt{Point} e sia la velocità che 
l'accelerazione, in quanto vettori, vengono rappresentati mediante la classe
\texttt{Vector}\footnote{Attenzione a non confodersi con i nomi della classi. 
Quella citata è più precisamente la classe \texttt{phy::Vector}, in quanto 
esiste anche la re-implementazione del vettore della libreria standar di \CC\ 
che invece è identificata dal \texttt{namespace nostd}}. Durante i test si è 
notato però che alcuni oggetti avevano necessità di memorizzare la loro 
posizione tramite stadi intermedi, senza quindi seguire forzatamente la 
rappresentazione intera della posizione fornita dalla classe \texttt{Point}. È 
quindi stata creata una ulteriore classe apposita identificata come 
\texttt{PrecisesPoint}. Non si è deciso di rimuovere completamente la precedente 
\texttt{Point} per una questione di ottimizzazione: fornendo una 
rappresentazione estremamente più precisa della posizione, la classe 
\texttt{PrecisesPoint} è necessariamente più pesane in memoria e più complessa 
da gestire. Dove quindi era possibile evitarla si è continuato a fare uso della 
semplice rappresentazione intera della posizione fornita dalla classe originale.
La classe \texttt{Body} è una semplice rappresentazione di un corpo puntiforme 
privo di massa con la capacità di muoversi nello spazio: incorpora quindi tutti 
i campi necessari per la posizione, velocità e accelerazione.

Le collisioni tra il giocatore e le piattaforme presenti nella mappa sono 
gestite sempre dalla libraria di fisica. In particolare sono calcolate partendo 
dalla posizione attuale del giocatore, successivamente si aggiorna la sua 
posizione in base alla sua velocità e accelerazione. Si esegue quindi un 
confronto tra la vecchia posizione e la nuova, controllando che nello 
spostamento il giocatore non abbia colliso con nessun oggetto. Nel caso però una
collisione si sia verificata, viene spostato il giocatore nel punto della 
collisione e vengono aggiornate velocità e accelerazione per rendere 
approssimativamente l'effetto del trasferimento di energia cinetica avvenuto 
durante la collisione, nonostante il calcolo dell'energia non sia effettivamente
eseguito.

\end{document}
